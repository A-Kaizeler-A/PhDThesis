% !TEX root = ../PhD Thesis.tex
\chapter{Introduction}
\pagenumbering{arabic}

\section{Stress Granules}

\subsection{What are Stress Granules?}

\Glspl{sg} are membraneless \gls{rnp} condensates, formed in response to a wide variety of stimuli\cite{shin_liquid_2017}, including hypoxia, nutrient deprivation, viral infection, T-cell activation, heat-shock, and oxidative stress\cite{anderson_stress_2015, anderson_stress_2009, mccormick_translation_2017, wallace_reversible_2015, curdy_proteome_2023, paget_stress_2023}. \glspl{sg} comprise aggregations of \gls{rna}, stalled pre-initiation complexes, and proteins\cite{anderson_stress_2009}, of which around half are \glspl{rbp}\cite{nunes_msgp_2019}. Characterised by rapid formation and swift disassembly upon cessation of the triggering stimuli (15 and 30 minutes with oxidative and heat-shock stress, respectively)\cite{hofmann_translation_2012}, \glspl{sg} are highly dynamic structures, with their protein constituents continuously transitioning into and out of them\cite{ivanov_stress_2019}.

\subsection{A brief history of Stress Granules}

Cytoplasmic aggregations induced by stress were first described in 1983 in \textit{Lycopersicon peruvianum} (Peruvian tomato plant), where exposure to heat-shock conditions (around 37–40 \gls{degree}) led to the aggregation of several types of heat-shock proteins\cite{nover_formation_1983}. These structures were termed heat-shock granules, named after the inducing stressor, and are often, albeit erroneously, cited as the first observation of \glspl{sg}. Subsequent studies demonstrated, however, that these plant granules do not contain \gls{rna} and therefore represent a different class of cytoplasmic assemblies\cite{weber_plant_2008}. This distinction does not imply that plants lack \gls{sg}; on the contrary, bona fide \glspl{sg} do form in plants\cite{weber_plant_2008}, concomitantly with heat-shock granules\cite{weber_plant_2008}. The first genuine \glspl{sg} were in fact observed in chicken only a few years later, in 1986\cite{collier_dynamic_1986}.
\\
Since then, \glspl{sg} have been observed to form across virtually all eukaryotic lineages, ranging from unicellular organisms such as \textit{Saccharomyces cerevisiae}\cite{hu_ssd1_2018} and \textit{Schizosaccharomyces pombe}\cite{zhang_fission_2025}, to classical model organisms including \textit{Drosophila melanogaster}\cite{van_leeuwen_identification_2022} and \textit{Caenorhabditis elegans}\cite{vidya_edc-3_2024}, in plants such as the aforementioned tomato species, \textit{Arabidopsis thaliana}\cite{chantarachot_dhh1ddx6-like_2020}, and \textit{Oryza sativa}\cite{zeng_viral_2025}, and in mammals including \textit{Mus musculus}\cite{namkoong_systematic_2018} and \textit{Homo sapiens}\cite{somasekharan_g3bp1-linked_2020}. 
\\
Such widespread evolutionary conservation is often taken as evidence of fundamental biological importance. However, this inference warrants caution. Numerous biological features persist across evolutionary time despite having little or no apparent functional relevance\cite{gould2010panda}. For instance, most extant whales retain rudimentary hind limb bones, vestigial structures that no longer contribute meaningfully to locomotion or survival\cite{bejder_limbs_2002}. Similarly, the vomeronasal organ is present in primates, including human fetuses, yet is functionally inactive (or missing) in adults\cite{won_vomeronasal_2000}. Although these examples are taxon-specific and far less ubiquitous than \glspl{sg} across eukaryotes, they illustrate a broader principle: persistence alone does not imply functional necessity.
\\
Additional examples include the widespread presence of pseudogenes and large fractions of so-called “junk \gls{dna}” in vertebrate genomes\cite{nishikimi_cloning_1994}. There is little evidence that evolution actively selected for these elements\cite{nishikimi_cloning_1994}. Rather, evolutionary dynamics are frequently dominated by negative selection, whereby traits are eliminated only when they confer a sufficiently deleterious effect. Classical examples include the rapid elimination of maladaptive phenotypes, such as the light-colored moths during the Industrial Revolution in London\cite{cook_peppered_2013}. Traits that are selectively neutral, or only mildly disadvantageous, may persist indefinitely simply because they are not strongly selected against\cite{nishikimi_cloning_1994, williams_identification_2016}.
\\
This principle is particularly evident in tumour evolution. Beyond the initial stages of oncogenic transformation, much of tumour growth proceeds under near-neutral evolutionary dynamics, with limited clonal selection except against cells harboring severely deleterious mutations\cite{williams_identification_2016}. This neutrality helps explain the extensive intratumoural heterogeneity observed in many cancers\cite{williams_identification_2016}. Analogous to mass extinction events in macroevolution\cite{marshall_forty_2023}, strong selective pressures are often episodic, becoming prominent only under extreme conditions,such as the introduction of therapeutic interventions\cite{williams_identification_2016}.
\\
Consequently, a biological feature may be conserved throughout evolution not because it confers a selective advantage, but because it does not impose a sufficient fitness cost to be eliminated. Conservation, therefore, does not necessarily imply positive selection. This argument is not intended to suggest that SGs lack biological function; indeed, their potential relevance in oncology is what motivated this work. Rather, it serves to emphasize that evolutionary conservation can at best support the hypothesis of functional importance, but cannot, on its own, constitute definitive proof.

\subsection{Stress Granule Assembly and Disassembly}

SG assembly is most commonly initiated by phosphorylation of \gls{eif2a}\cite{anderson_visibly_2002}, a central regulator of translation initiation\cite{anderson_visibly_2002}. Phosphorylation of this factor is sufficient to block translation initiation and consequently trigger \gls{sg} formation\cite{kedersha_rna-binding_1999}, and is mediated by four stress-responsive kinases that constitute the core of the \gls{isr}\cite{donnelly_eif2_2013, aulas_stress-specific_2017, wek_role_2018}: \gls{gcn2}, \gls{perk}, \gls{pkr}, and \gls{hri}. Each of these kinases is activated by distinct cellular insults and is required for an appropriate translational and granule-forming response to its corresponding stress\cite{aulas_stress-specific_2017, wek_role_2018}. For example, \gls{hri} is essential for \gls{sg} formation under oxidative stress\cite{mcewen_heme-regulated_2005}, but is dispensable in response to heat-shock\cite{aulas_stress-specific_2017}, whereas \gls{gcn2}, \gls{perk}, and \gls{pkr} are required, respectively, during \gls{aa} deprivation\cite{wek_histidyl-trna_1995}, perturbation of endoplasmic reticulum proteostasis\cite{harding_perk_2000}, and cellular stress\cite{srivastava_phosphorylation_1998} induced by viral infection, heat-shock, or ultraviolet irradiation. Consistent with this model, genetic defects in \gls{eif2a} itself\cite{kedersha_rna-binding_1999}, or pharmacological inhibition of the \gls{isr}\cite{sidrauski_small_2015}, markedly impairs \gls{sg} assembly.
\\
Phosphorylation of \gls{eif2a} inhibits the guanine nucleotide exchange factor \gls{eif2b}, resulting in global translational repression\cite{wek_role_2018}. This inhibition leads to ribosome run-off, polysome disassembly, and the accumulation of stalled pre-initiation complexes\cite{kedersha_dynamic_2000}. These translationally arrested \glspl{mrna}, together with associated translation initiation factors and \glspl{rbp}, form the core molecular scaffold of \glspl{sg}\cite{kedersha_dynamic_2000}. This nascent assembly subsequently recruits additional proteins through multivalent interactions, promoting further condensation and maturation of the granule\cite{kedersha_rna-binding_1999, kedersha_dynamic_2000}. Of central importance in this process are \gls{sg}-nucleating proteins such as \gls{g3bp1}, \gls{g3bp2}, \gls{tia1} and  \gls{tiar}\cite{aulas_stress-specific_2017}, which are also commonly used as \gls{sg} markers\cite{aulas_stress-specific_2017}.
\\
\Glspl{sg} can also assemble through \gls{eif2a}-phosphorylation independent mechanisms, arising from inhibition of alternative steps in translation initiation. An example of one such mechanism involves \gls{eif4a}, an \gls{atp}-dependent \gls{rna} helicase that is required for ribosome recruitment and for unwinding secondary structures within the \gls{5utr} of \gls{mrna}\cite{mazroui_inhibition_2006, kedersha_g3bpcaprin1usp10_2016}. Pharmacological inhibition of \gls{eif4a} activity, for example by hippuristanol or other compounds, impairs translation initiation and has been shown to robustly induce \gls{sg} formation in the absence of \gls{eif2a} phosphorylation\cite{mazroui_inhibition_2006, kedersha_g3bpcaprin1usp10_2016}.
\\
In addition to signaling-dependent mechanisms, physical changes in the intracellular environment can also promote \gls{sg} assembly\cite{bounedjah_macromolecular_2012}. Hypertonic stress leads to cellular shrinkage, resulting in an increased intracellular concentration of macromolecules and enhanced molecular crowding\cite{bounedjah_macromolecular_2012}. This elevated crowding strengthens weak, multivalent interactions among \gls{mrna}–protein complexes, thereby favouring their condensation and promoting \gls{sg} formation independently of canonical translation initiation signaling pathways\cite{bounedjah_macromolecular_2012}.
 
\begin{figure}[H]
	\centering
	\includegraphics[width=0.7\linewidth]{"../../../Desktop/Paper Images/TESE_Figu1"}
	\caption[Different Mechanisms Leading to SG Assembly]{\textbf{Different Mechanisms Leading to \gls{sg} Assembly} The schematic illustrates three mechanistically distinct pathways through which \glspl{sg} can assemble. \gls{eif2a} phosphorylation–independent \gls{sg} assembly (blue): In this pathway, \glspl{sg} form independently of \gls{eif2a} phosphorylation. One representative example is shown, in which inhibition of the RNA helicase \gls{eif4a} by hippuristanol impairs \gls{mrna} scanning and ribosome recruitment\cite{mazroui_inhibition_2006, kedersha_g3bpcaprin1usp10_2016}. This results in failure of cap-dependent translation initiation, leading to the accumulation of non-translating messenger ribonucleoprotein complexes and subsequent \gls{sg} assembly\cite{mazroui_inhibition_2006, kedersha_g3bpcaprin1usp10_2016}. \gls{eif2a} phosphorylation–dependent \gls{sg} assembly (red): In response to diverse cellular stress stimuli, one or more \gls{eif2a} kinases (\gls{perk}, \gls{pkr}, \gls{gcn2}, or \gls{hri}) are activated, resulting in phosphorylation of \gls{eif2a}\cite{donnelly_eif2_2013, aulas_stress-specific_2017, wek_role_2018}. This modification reduces ternary complex availability and causes global repression of translation initiation, constituting activation of the \gls{isr}) and promoting \gls{sg} assembly\cite{donnelly_eif2_2013, aulas_stress-specific_2017, wek_role_2018}. Molecular crowding–driven \gls{sg} assembly (green): \glspl{sg} can also assemble as a consequence of increased molecular proximity and macromolecular crowding. For example, hypertonic stress induces cell shrinkage, leading to elevated intracellular crowding that strengthens weak, multivalent interactions among \gls{mrna}–protein complexes\cite{bounedjah_macromolecular_2012}. This favors their condensation and promotes \gls{sg} formation independently of canonical translation initiation signaling pathways\cite{bounedjah_macromolecular_2012}.}
	\label{fig:sg_assembly}
\end{figure}

\noindent
Because \gls{sg} formation is intrinsically linked to polysome dynamics, pharmacological agents that disrupt polysomes tend to trigger or enhance \gls{sg} assembly, whereas compounds that stabilise polysomes inhibit their formation\cite{kedersha_mammalian_2007}. For example, edeine interferes with proper ribosome assembly and translation initiation, leading to polysome disassembly and robust induction of \glspl{sg}\cite{thomas_staufen_2005}. Similarly, puromycin induces premature termination of translation and promotes ribosome release from \glspl{mrna}, thereby destabilising polysomes and enhancing \gls{sg} formation\cite{kedersha_dynamic_2000,kedersha_mammalian_2007}. In contrast, cycloheximide stabilises polysomes by freezing elongating ribosomes on \glspl{mrna}, effectively preventing ribosome run-off and increasing the threshold of stress required to induce \gls{sg} assembly\cite{kedersha_dynamic_2000,kedersha_mammalian_2007}.
\\
\gls{sg} disassembly largely follows the reverse sequence of events that govern their assembly. Upon stress relief, \gls{eif2a} is dephosphorylated in the canonical pathway, leading to the restoration of translation initiation\cite{hofmann_molecular_2021, wheeler_distinct_2016}. Translation restart occurs in both canonical and non-canonical contexts and promotes the re-engagement of \glspl{mrna} with ribosomes, resulting in their reincorporation into polysomes\cite{hofmann_molecular_2021, wheeler_distinct_2016}. As translationally active \glspl{mrna} are withdrawn from \glspl{sg}, the number of stabilising intermolecular interactions within the granules decreases, ultimately driving their disassembly\cite{hofmann_molecular_2021, wheeler_distinct_2016}. During this process, \glspl{sg} progressively fragment into smaller assemblies, which are thought to be cleared through autophagy-dependent pathways\cite{wheeler_distinct_2016}, although the extent to which autophagy contributes to \gls{sg} turnover remains an active area of investigation\cite{ganassi_surveillance_2016, marcelo_stress_2021}.
\\
In contrast, considerably less is known about the disassembly mechanisms of \glspl{sg} formed  through increased molecular crowding. Notably, while \glspl{sg} assembled via both \gls{eif2a}-dependent and \gls{eif2a}-independent translational control pathways typically disassemble within approximately one hour following stress removal\cite{bounedjah_macromolecular_2012}, crowding-induced \glspl{sg} display markedly slower dynamics and can persist for several hours, with reported disassembly times extending up to nine hours\cite{bounedjah_macromolecular_2012}.

\subsection{Functions of Stress Granules}

A process conserved across all eukaryotes inevitably implies functional relevance. Even prior to the discovery of \glspl{sg}, it was well established that heat-shock and other forms of cellular stress induce a rapid arrest of translation\cite{storti_translational_1980, didomenico_heat_1982, nover_formation_1983}. The striking correlation between translational repression and \gls{sg} appearance naturally led to the assumption of a causal relationship between the two phenomena\cite{kedersha_rna-binding_1999}. Indeed, such causality does exist; however, it is not \glspl{sg} that impose translational arrest. Rather, translational arrest is the upstream event that drives \gls{sg} formation\cite{anderson_visibly_2002}.
\\
How did this confusion arise? Early observations showed that the RNA-binding proteins \gls{tia1} and \gls{tiar} bind untranslated \glspl{mrna} and localise to \glspl{sg}\cite{kedersha_rna-binding_1999}. Mutations in these proteins strongly impaired \gls{mrna} recruitment into \glspl{sg}, yet had no effect on the global sequestration of stress-induced untranslated transcripts\cite{kedersha_rna-binding_1999}. In parallel, although \gls{eif2a} phosphorylation was recognised as a key driver of translational arrest, cases were described in which translation inhibition occurred even in the absence of \gls{eif2a} phosphorylation\cite{duncan_protein_1989}, still accompanied by \gls{sg} formation\cite{kedersha_rna-binding_1999}. Because \glspl{sg} contained untranslated \glspl{mrna} and were present under conditions of translational blockade, even when canonical \gls{eif2a} signaling was bypassed, it was postulated that \glspl{sg} might play a central role in enforcing translation inhibition\cite{kedersha_rna-binding_1999}.
\\
However, subsequent work has demonstrated that \glspl{sg} are neither necessary nor sufficient for translational repression\cite{kedersha_g3bpcaprin1usp10_2016} and, in many contexts, have only modest effects on overall protein output\cite{mokas_uncoupling_2009}. Moreover, translation has been shown to occur within or in close association with \glspl{sg}, challenging the long-standing notion that \glspl{sg} function primarily as sites of \gls{mrna} storage to prevent translation\cite{mateju_single-molecule_2020}. Together, these findings indicate that \glspl{sg} are a consequence, rather than a cause, of stress-induced translational arrest\cite{mateju_stress_2022}, and that their functional role, if any, must extend beyond simple translational repression.
\\
As of the writing of this thesis, a growing hypothesis proposes that \glspl{sg} function as fine-tuners of the cellular stress response through selective sequestration\cite{marcelo_stress_2021}. \glspl{sg} have been reported to enhance cell survival by transiently sequestering specific pro-apoptotic\cite{arimoto_formation_2008, tsai_rhoarock1_2010, fujikawa_stress_2023} or senescence-associated factors, such as \gls{pai1} \cite{omer_stress_2018}, thereby limiting stress-induced senescence, and \textit{\gls{rb1}}, protecting it from degradation in an \gls{rbfox2}-dependent manner\cite{park_stress_2017}, among other examples.
\\
Nevertheless, although \gls{sg} assembly generally correlates with improved cellular fitness under stress\cite{paget_stress_2023}, \glspl{sg} do not appear to represent a strict life-or-death determinant, as cells deficient in \gls{sg} formation can still survive\cite{paget_stress_2023,matheny_rna_2021, matheny_corrigendum_2023}. The picture becomes even more ambiguous when selective transcript sequestration is considered. In most cases, \glspl{sg} recruit transcripts broadly, with the number of \gls{mrna} copies within \glspl{sg} largely reflecting their abundance in the total cellular transcriptome\cite{khong_stress_2017, matheny_transcriptome-wide_2019,matheny_rna_2021,matheny_corrigendum_2023}. One notable exception appears to be stress-induced transcripts\cite{somasekharan_g3bp1-linked_2020, glauninger_transcriptome-wide_2024}. Indeed, even in the earliest descriptions of "stress granules", it was hypothesised that \glspl{sg} might function to prioritise stress-responsive gene expression, as translation under stress conditions is largely restricted to these transcripts\cite{nover_formation_1983}.
\\
From this perspective, transcript selectivity may not arise from the preferential recruitment of specific \glspl{mrna} into \glspl{sg}, but rather from the sequestration of the bulk of non-stress transcripts\cite{glauninger_transcriptome-wide_2024}, effectively excluding stress-induced \glspl{mrna} from \glspl{sg} and allowing their continued translation\cite{glauninger_transcriptome-wide_2024}.
\\
This model of selective sequestration is, however, directly challenged, at least at the transcript level as previously mentioned, by observations showing that \glspl{mrna} recruited to \glspl{sg} can undergo translation at rates comparable to those of freely diffusing transcripts under non-stressed conditions\cite{mateju_single-molecule_2020}. Such findings undermine the notion that transcript sequestration by \glspl{sg} constitutes a dominant or exclusive mechanism for translational regulation\cite{glauninger_stressful_2022}.
\\
In light of this apparent functional ambiguity, a fundamental question emerges: why are \glspl{sg} still attributed such a wide range of cellular functions, and why are they so frequently implicated in diverse pathological contexts? Part of the answer may lie not in definitive mechanistic evidence, but in their conspicuous and reproducible presence. \glspl{sg} are readily observable, form robustly across cell types and stresses, and appear whenever translation is perturbed. Their persistence and ubiquity make them natural focal points for hypothesis generation.
\\
This phenomenon reflects a broader tendency in biological research: highly visible and evolutionarily conserved structures are rarely assumed to be functionless. \glspl{sg} are consistently “there” (as George Mallory would put it), forming reliably under conditions of translational stress, and their very existence invites functional attribution. As a result, they have become central players in numerous models of stress adaptation and disease, even as definitive evidence for many proposed roles remains incomplete\cite{glauninger_stressful_2022}. In this sense, the prominence of \glspl{sg} in both physiological and pathological narratives may reflect not only their biological relevance, but also the interpretive weight we assign to persistent, conspicuous cellular structures whose functions are still being actively defined.

\subsection{Stress Granules in Disease}

As anticipated in the previous section, \glspl{sg}, like most fundamental cellular processes, have been extensively implicated in disease. \glspl{sg} are cytoplasmic assemblies characterised by the aggregation of \glspl{mrna} and proteins, approximately half of which are \glspl{rbp}\cite{nunes_msgp_2019}. Many \glspl{rbp} contain \glspl{idd}\cite{wear_proteins_2015, marcelo_stress_2021}, which confer structural flexibility but also predispose these proteins to aberrant self-association and aggregation\cite{wear_proteins_2015, marcelo_stress_2021}. Notably, this same class of proteins is prominently implicated in neurodegenerative disorders\cite{marcelo_stress_2021}. For example, members of the \gls{atxn} protein family, whose dysfunction underlies several forms of \gls{sca}, exhibit aggregation-prone properties linked to disease pathology\cite{bevivino_expanded_2001, elden_ataxin-2_2010}.
\\
Given that both \gls{sg} biology and many neurodegenerative diseases centrally involve protein aggregation, it was perhaps inevitable that mechanistic connections between the two would be proposed. Indeed, numerous studies have suggested that alterations in \gls{sg} dynamics, particularly defects in assembly, maturation, or disassembly, may promote pathological aggregation\cite{mackenzie_tia1_2017}. One prevailing hypothesis posits that the insoluble aggregates characteristic of several neurodegenerative disorders may originate from \glspl{sg} that fail to properly disassemble, thereby transitioning from transient, reversible assemblies into persistent, cytotoxic inclusions\cite{dobra_relation_2018, wolozin_stress_2019, marcelo_stress_2021, wang_targeting_2020}.
\\
\glspl{sg} have also been extensively implicated in antiviral responses, a topic that has been reviewed in detail elsewhere\cite{guan_multiple_2023}. Many viruses induce \gls{sg} formation, most commonly through activation of \gls{pkr}\cite{guan_multiple_2023, lindquist_activation_2011, choi_polyhexamethylene_2022}, leading to phosphorylation of \gls{eif2a}, although \gls{perk}-mediated \gls{eif2a} phosphorylation has also been reported in certain contexts\cite{guan_multiple_2023, zhou_porcine_2017}. As discussed in the sections on \gls{sg} assembly and disassembly, these signaling events position \gls{sg} formation downstream of translational inhibition.
\\
The functional consequences of \gls{sg} assembly during viral infection, however, remain ambiguous. In some cases, \gls{sg} formation correlates with impaired viral replication, consistent with a role in host defense\cite{albornoz_stress_2014}. In other systems, \gls{sg} induction appears to have little or no measurable impact on viral propagation\cite{lindquist_activation_2011}. On the other hand, they have also been found to be hijacked by viruses, having their assembly linked to increased viral replication\cite{sun_newcastle_2017}.
\\
This variability suggests that the relationship between \glspl{sg} and viral replication is at least partly context-dependent, influenced by viral species, host cell type, and the specific mechanisms by which viruses interact with the host translational machinery.
\\
Perhaps more striking than \gls{sg} induction is the remarkable diversity of viral strategies that actively impair \gls{sg} formation\cite{guan_multiple_2023}. Viruses across essentially all major classes—including \gls{ssrna} positive- (\gls{merscov}\cite{nakagawa_inhibition_2018}, \gls{sarscov2}\cite{nabeel-shah_sars-cov-2_2022} and Zika virus\cite{bonenfant_zika_2019}) and negative-sense (Influenza A virus\cite{khaperskyy_influenza_2014}) viruses, \gls{dsrna} (Pseudorabies virus\cite{xu_pseudorabies_2020}), \gls{dsdna} viruses (Kaposi's Sarcoma-associated herpesvirus\cite{sharma_kshv_2017}), and retroviruses (\gls{hiv1}\cite{abrahamyan_novel_2010}), have evolved mechanisms to interfere with \gls{sg} assembly. Some viruses achieve this by promoting \gls{eif2a} dephosphorylation\cite{xu_pseudorabies_2020} or by inhibiting upstream kinases such as \gls{pkr} or \gls{perk}\cite{nakagawa_inhibition_2018, khaperskyy_influenza_2014, sharma_kshv_2017}, thereby preventing translational arrest and the downstream formation of \glspl{sg}. Others employ more direct strategies, including the cleavage, sequestration, or functional inhibition of core \gls{sg}-nucleating proteins such as \gls{g3bp1}\cite{nabeel-shah_sars-cov-2_2022, bonenfant_zika_2019} or \gls{tia1}\cite{emara_interaction_2007}.
\\
In several cases, viral inhibition of \gls{sg} formation is associated with enhanced viral replication\cite{nakagawa_inhibition_2018, nabeel-shah_sars-cov-2_2022, bonenfant_zika_2019}, supporting the view that \glspl{sg} predominantly exert antiviral functions. This observation has led to the proposal that certain viruses actively antagonize \glspl{sg} to evade host defenses\cite{nakagawa_inhibition_2018,guan_multiple_2023}, whereas others may tolerate or even exploit \gls{sg} assembly\cite{lindquist_activation_2011, sun_newcastle_2017, guan_multiple_2023}. Nonetheless, such interpretations warrant caution. Many viral strategies that inhibit \gls{sg} formation act at very upstream levels of cellular regulation, for example, by preventing \gls{eif2a} phosphorylation\cite{xu_pseudorabies_2020}, an event central to the \gls{isr} and to global translational control\cite{donnelly_eif2_2013, aulas_stress-specific_2017, wek_role_2018}. It is therefore plausible that these viruses evolved primarily to suppress the \gls{isr} as a whole, with \gls{sg} inhibition arising as a downstream consequence rather than as a direct selective target.
\\
More compelling evidence for a direct antiviral role of \glspl{sg} comes from viruses that specifically target \gls{sg}-nucleating \glspl{rbp} such as \gls{g3bp1} or \gls{tia1}\cite{nabeel-shah_sars-cov-2_2022, bonenfant_zika_2019, emara_interaction_2007}. These proteins are less universally essential than the \gls{isr} itself, lending support to the idea that \glspl{sg}, or at least \gls{sg}-associated functions, pose a direct obstacle to viral replication. Interestingly, viruses that directly target these \gls{sg} components appear to be enriched among positive-sense \gls{ssrna} viruses\cite{guan_multiple_2023}. A possible explanation is that this viral class also infects the broadest range of eukaryotic hosts\cite{koonin_origins_2015}. From an evolutionary perspective, it would be informative to determine when, phylogenetically, viral strategies shifted from targeting the \gls{isr} globally to directly antagonising \gls{sg} components, an adaptation that likely emerged after the evolution of eukaryotes. However, given that most viral lineages are thought to have evolved independently from prokaryotic viruses (brilliant reviewed elsewhere\cite{koonin_origins_2015}), the repeated emergence of \gls{sg}-targeting mechanisms instead points toward convergent evolution. This convergence, in turn, supports the notion that \glspl{sg} represent a meaningful barrier to viral replication. Nevertheless, disentangling \gls{sg}-specific antiviral effects from the broader \gls{rna}-regulatory functions of these \glspl{rbp} remains a significant challenge in defining the precise contribution of \glspl{sg} to antiviral defense\cite{glauninger_stressful_2022}.
\\
\glspl{sg} have also been implicated in cellular senescence and, by extension, in ageing. They have been reported to counteract senescence by selectively sequestering \gls{pai1}\cite{omer_stress_2018}, a well-established promoter of cell-cycle arrest and a key component of the \gls{sasp}\cite{omer_stress_2018}. While cellular senescence is generally considered beneficial in the context of tumour suppression, persistent senescence is detrimental and contributes to tissue dysfunction\cite{martins-silva_exploring_2025}. Moreover, the \gls{sasp} itself can paradoxically promote tumour progression\cite{martins-silva_exploring_2025}, a duality that mirrors the context-dependent roles attributed to \glspl{sg} in cancer biology\cite{anderson_stress_2015}.
\\
Given that both \glspl{sg} and senescence are induced by cellular stress\cite{omer_g3bp1_2020}, it is tempting to conceptualise them as alternative, and potentially antithetic, stress responses that are balanced according to cellular context and stress severity. Supporting this notion, senescent cells have been shown to suppress \gls{sg} formation through depletion of key \gls{sg}-nucleating \glspl{rbp}, \gls{g3bp1}, \gls{tia1} and \gls{tiar}, as well as their transcription factor \gls{sp1}\cite{moujaber_dissecting_2017}. As a result, \gls{sg} assembly is impaired even under conditions of robust \gls{eif2a} phosphorylation, which is commonly observed during senescence\cite{moujaber_dissecting_2017}. This finding suggests that senescence actively enforces a cellular state refractory to \gls{sg} formation\cite{omer_stress_2018, moujaber_dissecting_2017}.
\\
However, this relationship appears to be more complex and context-dependent than a simple antagonism. Notably, \gls{g3bp1} has been reported to be required for the establishment of the \gls{sasp}\cite{omer_g3bp1_2020}, placing it functionally upstream of a hallmark senescence program, countering its reported depletion in senescent cells\cite{moujaber_dissecting_2017}. Importantly, depletion of \gls{g3bp1} does not abolish senescence itself but selectively impairs \gls{sasp} expression\cite{omer_g3bp1_2020}, indicating that senescence maintenance and \gls{sasp} execution can be uncoupled\cite{omer_g3bp1_2020}. One speculative interpretation is that \glspl{sg} may represent an early or transient attempt by the cell to buffer stress and avoid irreversible senescence. Failure of this adaptive response could then favor commitment to senescence and the emergence of the \gls{sasp}, ultimately promoting chronic inflammation. While this model remains speculative, it highlights a potentially dynamic interplay between \gls{sg} biology and senescence programs that warrants further investigation.
\\
Beyond the roles discussed above, and some others not addressed at length (inflammation\cite{paget_stress_2023}) \glspl{sg} have also been extensively implicated in cancer biology\cite{anderson_stress_2015}. Given the breadth and complexity of their functions in tumourigenesis and cancer progression, this topic is addressed separately and in greater detail in Chapter 2.

\section{RNA-Binding Proteins}

\subsection{What are RNA-Binding Proteins?}

\glspl{rbp} are proteins that interact with \gls{rna} molecules in both single-stranded and double-stranded conformations\cite{corley_how_2020}. Far from representing a specialised or recently evolved protein class, \glspl{rbp} are thought to be among the most ancient functional proteins\cite{kerner_evolution_2011, farias_rnp-world_2022, gerstberger_census_2014}. Current models of early molecular evolution suggest that the first biological systems were not purely protein-based, but instead relied on intimate cooperation between \gls{rna} and peptides\cite{farias_rnp-world_2022}. While the classical \gls{rna} world hypothesis posits \gls{rna} as the sole primordial biopolymer, more recent views favor an \gls{rna}–peptide world, in which short peptides interacted with structured \glspl{rna} to stabilise them and enhance catalytic efficiency\cite{farias_rnp-world_2022}. In this context, primitive \gls{rna}-binding peptides associated with proto-ribosomal structures, are thought to have facilitated the emergence of early metabolism by conferring structural stability and functional versatility to \gls{rna}\cite{farias_rnp-world_2022, hentze_brave_2018}.
\\
As biological complexity increased, the repertoire of proteins expanded and diversified\cite{farias_rnp-world_2022}. Many newly evolved proteins gradually lost \gls{rna}-binding capacity as they specialised toward enzymatic, structural, or signaling roles\cite{farias_rnp-world_2022, gerstberger_census_2014}. Nonetheless, a substantial fraction of modern proteins retained \gls{rna}-binding functions, reflecting the central role of \gls{rna} in gene expression and cellular regulation\cite{farias_rnp-world_2022, hentze_brave_2018, gerstberger_census_2014}. Today, \glspl{rbp} constitute a large and functionally diverse class of proteins that expanded broadly in eukaryotes\cite{corley_how_2020, holmqvist_rna-binding_2018} and that govern virtually every aspect of \gls{rna} metabolism, and as such, cellular function and homeostasis\cite{corley_how_2020, holmqvist_rna-binding_2018, gerstberger_census_2014, hentze_brave_2018, kelaini_rna-binding_2021, van_nostrand_large-scale_2020}.
\\
At the molecular level, \glspl{rbp} recognise \gls{rna} through a variety of conserved \gls{rna}-binding domains, such as \glspl{rrm}, \glspl{kh}, \glspl{zf}, and DEAD-box helicase domains, as well as through \glspl{idd} that enable dynamic and multivalent interactions\cite{corley_how_2020, gerstberger_census_2014}. The prevalence of \glspl{idd} among \glspl{rbp} endows them with structural plasticity, allowing rapid assembly and disassembly of \gls{rnp} complexes and facilitating the formation of higher-order \gls{rna}–protein assemblies, including membraneless organelles such as \glspl{sg} and \glspl{pb}\cite{hentze_brave_2018, redding_stress_2023, wear_proteins_2015, marcelo_stress_2021}.
\\
Given their central role in post-transcriptional regulation, \glspl{rbp} are critical determinants of cellular identity and adaptability\cite{holmqvist_rna-binding_2018, kelaini_rna-binding_2021, babitzke_regulation_2009}. Perturbations in \gls{rbp} function have been linked to a wide range of pathological conditions, including neurodegenerative diseases, cancer, viral infections, and ageing\cite{kelaini_rna-binding_2021, redding_stress_2023, wear_proteins_2015, marcelo_stress_2021}. Understanding the fundamental biology of \glspl{rbp} is therefore essential for elucidating how cells integrate \gls{rna} regulation with stress responses and how dysregulation of these processes can contribute to disease.

\subsection{Functions of RNA-Binding Proteins}

As outlined above, \glspl{rbp} act as master regulators of \gls{rna} biology, exerting control over virtually every stage of the \gls{rna} life cycle. Their functions span all major \gls{rna} classes, including \gls{rrna}, \gls{trna}, \gls{mrna}, \gls{lncrna}, and other regulatory \gls{rna} species\cite{gerstberger_census_2014, kelaini_rna-binding_2021}. Through direct and dynamic interactions with these \glspl{rna}, \glspl{rbp} orchestrate \gls{rna} synthesis, processing, modification, transport, localisation, translation, and degradation\cite{gerstberger_census_2014, kelaini_rna-binding_2021}.  
\\
\gls{nelfe} binds to transcripts of the proto-oncogene \textit{\gls{myc}}, stabilising the \gls{mrna} and promoting its translation, thereby contributing to tumour progression\cite{kelaini_rna-binding_2021}. \gls{hurelav} interacts with \glspl{are} within the \gls{3utr} of target \gls{mrna}, leading to their stabilisation, particularly for transcripts involved in inflammatory responses\cite{kelaini_rna-binding_2021}. In contrast, \gls{ttp} recognises similar \gls{are} on overlapping sets of transcripts but promotes their deadenylation and degradation instead\cite{kelaini_rna-binding_2021}.
Beyond their direct effects on \gls{rna} stability, these examples illustrate how \glspl{rbp} constitute an additional regulatory layer within signaling pathways, fine-tuning gene expression outcomes by selectively stabilising or destabilising specific \gls{rna} populations.
\\
Another clear example of this functional dichotomy is found in the role of RNA-binding proteins in translational control. \glspl{pabp} enhance cap-dependent translation by stabilising pre-initiation complexes\cite{harvey_trans-acting_2018}. In contrast, \glspl{cpeb} generally function as translational repressors, although their activity can be modulated in response to specific cellular cues\cite{harvey_trans-acting_2018}. \gls{larp1} exemplifies the context-dependent nature of \gls{rbp} function, as it has been shown to both repress and promote translation depending on cellular conditions and signaling status\cite{harvey_trans-acting_2018, burrows_rna_2010, fonseca_-related_2015}.
\\
\glspl{rbp} also play central roles in \gls{rna} editing. The most prominent example is the \gls{adar} family of enzymes, which bind \gls{dsrna} and catalyse the deamination of adenosine to inosine\cite{savva_adar_2012}. This form of \gls{rna} editing is particularly prevalent in the nervous system, where it is essential for proper neuronal development and function\cite{savva_adar_2012}.
\\
Beyond editing and stability control, \glspl{rbp} are already critically involved at the earliest stages of \gls{rna} maturation. Proteins such as \gls{cpsf30} and \gls{wdr33} are required for 3'-end processing and polyadenylation\cite{liu_non-canonical_2024}, while \gls{rngtt} and \gls{rnmt} are essential for the formation of the 5' cap structure\cite{liang_rna_2023}. Both polyadenylation and capping are prerequisite steps for \gls{mrna} stability, translational competence, and subsequent nuclear export\cite{van_nostrand_large-scale_2020}. Export to the cytoplasm is mediated by \glspl{rbp} such as \gls{tapnxfw} and its cofactor \gls{p15nxt1}\cite{aibara_principal_2015}.
\\
In addition to their role in nuclear export, \glspl{rbp} are also key determinants of \gls{rna} localisation within the cytoplasm, ensuring the spatial regulation of gene expression\cite{gerstberger_census_2014, kelaini_rna-binding_2021, van_nostrand_large-scale_2020}. In the context of this thesis, \glspl{rbp} are of particular relevance to \gls{sg} biology, where they constitute more than half of the protein components\cite{nunes_msgp_2019}. \gls{sg}-nucleating \glspl{rbp} such as \gls{g3bp1}/\gls{g3bp2} and \gls{tia1}/\gls{tiar} are essential for proper stress granule assembly\cite{aulas_stress-specific_2017}, underscoring the central role of \glspl{rbp} in the formation and regulation of these dynamic \gls{rnp} assemblies.

\subsection{RBPs and Splicing}

One of the most fundamental functions of \glspl{rbp} is their role in pre-\gls{mrna} splicing\cite{van_nostrand_large-scale_2020}. First discovered in 1977 because of adenovirus infection\cite{chow_amazing_1977, berget_spliced_1977}, splicing is a defining feature of eukaryotic gene expression and is particularly expanded in higher eukaryotes such as mammals, where it occurs in circa 95\% of genes and greatly increases proteomic diversity\cite{marasco_physiology_2023}. Through alternative splicing, different combinations of exons can be joined from a single gene, allowing the production of multiple protein isoforms from a single genomic locus\cite{marasco_physiology_2023}. Together with the use of alternative promoters\cite{martins-silva_exploring_2025, marasco_physiology_2023}, alternative splicing represents a principal mechanism by which the functional breadth of the genome is expanded beyond gene number alone.
\\
\glspl{rbp} are essential for the assembly, regulation, and function of the spliceosome, a large and highly dynamic \gls{rnp} complex responsible for intron removal\cite{marasco_physiology_2023}. Beyond constitutive splicing, \glspl{rbp} also regulate alternative splicing by promoting or inhibiting the inclusion of specific exons, thereby shaping transcript identity in a cell type– and context-dependent manner\cite{van_nostrand_large-scale_2020, marasco_physiology_2023, tapial_atlas_2017}. By binding to cis-acting elements within pre-\glspl{mrna}, \glspl{rbp} can either enhance or repress splice site recognition, influencing both exon inclusion and exclusion\cite{van_nostrand_large-scale_2020, marasco_physiology_2023, tapial_atlas_2017}.
\\
As a substantial fraction of splicing occurs co-transcriptionally\cite{van_nostrand_large-scale_2020, marasco_physiology_2023}, many \glspl{rbp} function in close physical and temporal proximity to the transcription and splicing machiner\cite{van_nostrand_large-scale_2020, marasco_physiology_2023}. Because splicing takes place predominantly within the nucleus, \glspl{rbp} involved in this process must be efficiently localised to the nuclear compartment\cite{van_nostrand_large-scale_2020, marasco_physiology_2023}. Consequently, following their synthesis in the cytoplasm, these proteins rely on regulated nuclear import mechanisms to reach the nucleus\cite{guo_nuclear-import_2018}, where they execute their splicing-related functions.
\\
How do \glspl{rbp} exert control over alternative splicing decisions? In most cases, they do so by binding directly to specific sequence or structural elements within pre-\glspl{mrna}, although indirect regulation through interactions with spliceosomal components can also occur\cite{van_nostrand_large-scale_2020}. Depending on both the identity of the \gls{rbp} and the position of its binding site relative to splice sites, this binding can either promote or inhibit intron removal or alternative exon inclusion\cite{van_nostrand_large-scale_2020, tapial_atlas_2017}.
\\
A useful conceptual framework to describe this positional logic is provided by \gls{rna} splicing maps (Figure 2). Splicing maps are schematic representations that integrate \gls{rbp} binding positions along a pre-\gls{mrna} with the corresponding splicing outcomes\cite{van_nostrand_large-scale_2020}. They depict where an \gls{rbp} binds, such as within an exon, upstream or downstream introns, or near splice junctions—and indicate whether binding at each position enhances or represses exon inclusion\cite{van_nostrand_large-scale_2020}. Importantly, splicing maps reveal that the same \gls{rbp} can exert opposite effects depending on binding location\cite{van_nostrand_large-scale_2020}. For example, binding downstream of an alternative exon may promote its inclusion, whereas binding within the exon or upstream intron may favor exon skipping\cite{van_nostrand_large-scale_2020}.
 
\begin{figure}[H]
	\centering
	\includegraphics[width=0.7\linewidth]{"../../../Desktop/Paper Images/RBFOX2"}
	\caption[RNA binding map of HepG2 and K562 cells exposed to shRNA against RBFOX2]{\textbf{\gls{rna} binding map of HepG2 and K562 cells exposed to \gls{shrna} against \gls{rbfox2}} (Top) Schematic of an alternative exon skipping event, regulated by an \gls{rbp}, \gls{rbfox2}. (Middle) Normalised binding of \gls{rbfox2} to metagenomic regulatory sequences of alternative exons. The plot discriminates excluded (blue), included (red) or maintained (black) alternative exons upon \gls{rbfox2} knockdown, obtained from \gls{encode}\cite{the_encode_project_consortium_integrated_2012, luo_new_2020}. The total number of events in each category can be found beneath the plot in brackets, preceded by the number of events for which there is at least one binding site for that \gls{rbp}. (Bottom) \gls{fdr} of excluded (blue) or included (red) normalised binding, in relation to maintained events. Vertical lines delimitate the following metagenomic splicing regulatory sequences, in order: last 50 \glspl{nt} of upstream constitutive exon; first 200 \glspl{nt} of upstream intron; last 200 \glspl{nt} of upstream intron; first 50 \glspl{nt} of alternative exon; last 50 \glspl{nt} of alternative exon; first 200 \glspl{nt} of downstream intron; last 200 \glspl{nt} of downstream intron; first 50 \glspl{nt} of downstream constitutive exon. This example shows normalised \gls{rbfox2} binding enriched downstream of alternative exon. Upon \gls{rbfox2} knockdown, this binding is therefore lost, consistent with increased exon skipping. These results align with the known role of \gls{rbfox2} in promoting exon inclusion\cite{van_nostrand_large-scale_2020}, providing a proof of concept for this approach.}
	\label{fig:rbfox2}
\end{figure}

\noindent
These maps are typically derived from the integration of transcriptome-wide binding data (e.g., \gls{clip}-based approaches) with splicing outcome measurements (e.g., \gls{rnaseq})\cite{van_nostrand_large-scale_2020}. By providing a spatial and functional overview of \gls{rbp} activity, \gls{rna} splicing maps offer a powerful way to predict and rationalise how \glspl{rbp} regulate alternative splicing across different transcripts and cellular contexts\cite{van_nostrand_large-scale_2020}.
\\
Well-characterised examples illustrate how positional binding of \glspl{rbp} determines alternative splicing outcomes. \gls{rbfox2} and \gls{qki} generally promote inclusion of alternative exons when binding to intronic regions downstream of the exon\cite{van_nostrand_large-scale_2020}. In contrast, \gls{ptbp1} typically promotes exon skipping by binding to upstream intronic elements flanking alternative exons\cite{van_nostrand_large-scale_2020, georgilis_ptbp1-mediated_2018}. Other \glspl{rbp}, such as \gls{prpf8}, can directly lead to alternative exon inclusion by biding near to it, or to exclusion by binding to the 5' of the upstream intron\cite{van_nostrand_large-scale_2020}. These positional effects were defined through the integration of \gls{clip}-based approaches with \gls{rna} sequencing analyses following experimental perturbation of \gls{rbp} function, most commonly through knockdown strategies such as \gls{shrna}-mediated depletion\cite{van_nostrand_large-scale_2020, georgilis_ptbp1-mediated_2018}.
\\
Importantly, \gls{rna} splicing maps are not limited to describing the effects of \gls{rbp} depletion alone. Once established, they provide a predictive framework for interpreting splicing changes arising from any perturbation that alters \gls{rbp} activity\cite{georgilis_ptbp1-mediated_2018}. Such perturbations include chemical inhibition, altered expression levels, disruption of interacting partners within the same regulatory complex, post-translational modifications that affect binding or activity, or changes in subcellular localization\cite{georgilis_ptbp1-mediated_2018}. For instance, sequestration of \glspl{rbp} into stress granules effectively removes them from the nucleus\cite{guo_nuclear-import_2018}, thereby impairing their ability to regulate splicing despite unchanged expression levels.
\\
Altogether, understanding how a given \gls{rbp} binds \gls{rna} and modulates splicing decisions opens new avenues for targeted manipulation of alternative splicing. This strategy has gained increasing interest in disease contexts such as cancer, where specific splice isoforms contribute to tumorigenesis and disease progression\cite{rogalska_regulation_2023}. Modulating \gls{rbp} activity to shift splicing patterns can therefore represent a therapeutic opportunity, either by suppressing oncogenic isoforms or by promoting intron retention to generate neoantigens that enhance anti-tumor immune responses\cite{rogalska_regulation_2023}.

\section{Mitochondria}

\subsection{What is a Mitochondrion?}

Perhaps the most defining organelle of eukaryotic cells, arguably even more central than stress granules, despite the plethora of controversial functions attributed to them, the mitochondrion is a hallmark of eukaryotic life\cite{casanova_mitochondria_2023}. Although mitochondria are often described as existing in a symbiotic relationship with the nucleus, this interaction is more accurately characterised as an obligate mutualism, in which both partners are fundamentally dependent on one another for survival\cite{zachar_endosymbiosis_2020}.
\\
A useful analogy is that of figs and fig wasps: specific wasp species are the exclusive pollinators of particular fig species, while the wasps themselves depend on the figs for shelter and nourishment\cite{segar_diversity_2025}. In such systems, neither organism can persist independently. In the case of mitochondria, this interdependence is even more pronounced. The extensive functional overlap between mitochondrial and cellular processes renders the relationship inseparable, to the point that the mitochondrion can be viewed as simultaneously occupying the roles of both partners in the mutualism\cite{zachar_endosymbiosis_2020}.
\\
Nevertheless, they are technically not true obligate mutualists, as some rare and highly specialised exceptions of eukaryotic cells lacking mitochondria have been described (perhaps the most famous being the oxymonad Monocercomonoides\cite{karnkowska_eukaryote_2016}), and even though mitochondria themselves usually cannot live independently, some such cases have also been reported\cite{stephens_characterization_2020}, albeit still controversial.
\\
Mitochondria were first observed in 1857 by Albert von Kölliker and initially described as intracellular granules\cite{andrieux_mitochondria_2021, harrington_mitochondria_2023}. Over the following decades, they underwent several nomenclatural revisions, being referred to as sarcosomes and bioblasts, before Carl Benda coined the term mitochondria in 1898, derived from the Greek mitos (thread) and chondrion (granule)\cite{andrieux_mitochondria_2021, harrington_mitochondria_2023}. The defining functional characterisation of these double-membrane organelle as the “powerhouse of the cell” emerged only much later in 1957\cite{siekevitz_powerhouse_1957}, but has remained popular ever since, reflecting how their importance is deeply embedded within cellular physiology. 

\subsection{The Mitochondrial Genome}

Mitochondria are widely thought to originate from an endosymbiotic event in which a proto-eukaryotic cell engulfed a prokaryotic organism, establishing a long-term association that ultimately proved beneficial to both partners\cite{harrington_mitochondria_2023, andrieux_mitochondria_2021, casanova_mitochondria_2023, zachar_endosymbiosis_2020}. Multiple lines of evidence support this endosymbiotic theory, most notably the extensive similarities between mitochondria and bacteria:
\\
Mitochondria cannot be generated\textit{de novo}; instead, they arise exclusively through the growth and binary fission of pre-existing mitochondria, a mode of replication reminiscent of bacterial division\cite{kraus_function_2021}. Their membranes share key features with bacterial membranes, including the presence of porins in the outer membrane\cite{takeda_oligomer-based_2025} and cardiolipin in the inner membrane\cite{geiger_multiple_2023}. Furthermore, mitochondria possess their own ribosomes, which are structurally and functionally more similar to bacterial ribosomes than to their eukaryotic cytosolic counterparts\cite{seely_mechanisms_2022}.
\\
Perhaps the strongest support for an endosymbiotic origin lies in the mitochondrial genome. Mitochondria retain their own DNA, which is distinct from nuclear DNA and closely resembles bacterial genomes in both sequence and organisation\cite{geiger_multiple_2023}. Phylogenetic analyses indicate that mitochondria most likely evolved from an alphaproteobacterial ancestor\cite{geiger_multiple_2023}. Consistent with this ancestry, mitochondrial DNA is circular in most species, in contrast to the linear chromosomes of the eukaryotic nucleus\cite{geiger_multiple_2023}. While linear mitochondrial genomes do exist in certain taxa and exhibit unique features\cite{stampar_linear_2019}, this thesis will focus on the circular mitochondrial genome, which is present in the vast majority of bilaterian organisms\cite{stampar_linear_2019}.
\\
As most bilaterians, the human mitochondrial genome is a circular, \gls{dsdna} molecule of 16,569 \glspl{bp}, encoding 37 genes with no introns\cite{habbane_human_2021}. The two strands are asymmetrical: the heavy (H) strand, rich in guanine and adenine, contains the majority of the genes, while the light (L) strand, enriched in cytosine and thymine, encodes a smaller fraction\cite{habbane_human_2021}. CITATION HERE ON THE FOLLOWING PARAGRAPH, DONT FORGET. This genome also has the particularity of being solely maternal in origin (although some controversial studies exist), and of having higher mutation rates than the nuclear genome, although the reasons for this are unclear, with some defending that this is due to the higher proximity to reactive oxygen species, but conflicting data exists. This exclusive origin, with the higher mutation rates, make mitochondria genome invaluable in human evolutionary analyses and forensic studies.
\\
Transcription in mitochondria is carried out by a DNA-dependent RNA polymerase, \gls{polrmt}, but efficient initiation of transcription requires the coordinated action of \gls{polrmt} with \gls{tfam} and \gls{tfb2m}\cite{dsouza_mitochondrial_2018}. Transcription occurs bidirectionally from promoters on both the H and L strands, producing long polycistronic primary transcripts that include \gls{rrna}, protein-coding genes, and \gls{trna}\cite{dsouza_mitochondrial_2018}. A unique feature of mitochondrial transcription is the generation of \gls{dsrna} molecules due to overlapping and bidirectional transcription of the circular genome\cite{lopez-polo_release_2024}.
\\
Maturation of these transcripts largely follows the \gls{trna} punctuation model, whereby mt-\glspl{trna} interspersed between coding sequences are excised by endonucleases, releasing mature \glspl{mrna} and \glspl{rrna}\cite{dsouza_mitochondrial_2018}. While this model explains the processing of most transcripts, not all primary transcript cleavage events are flanked by \glspl{trna}, indicating additional processing mechanisms remain to be fully characterised\cite{dsouza_mitochondrial_2018}.
\\
Following cleavage, mitochondrial \glspl{mrna} undergo post-transcriptional modification. Although almost all mitochondrial \glspl{mrna} are polyadenylated (with the exception of MT-ND6), unlike nuclear \glspl{mrna}, they are not canonically capped by guanine. While most mitochondrial transcripts are not capped, a small fraction is capped by \gls{nadh} instead\cite{bird_highly_2018}.
\\
Despite their own genome and strategies to transcribe it, proper mitochondrial function does require some nuclear encoded transcripts/proteins as well \cite{harrington_mitochondria_2023, casanova_mitochondria_2023}.

\subsection{Functions of the Mitochondrion}

As an intrinsic part of eukaryote cells, mitochondria are likewise integral for many of their functions. Generally, these can be divided into two main branches. One related to energy production, and another one has a central hub of regulatory pathways. 
\\
In terms of energy production, their main role is in producing Adenosine triphosphate (ATP), the main "energy molecule of the cell". They do this through a process called mitochondrial oxidative phosphorylation (OXPHOS), in which 36 moleculas of ATP are produced per unit of glucose. In comparison, perhaps the most common alternative is cytosolic aerobic glycolysis, in which only 2 molecules of ATP are produced per glucose, a process in which vastly less efficient in pure energetic terms. 
\\
A particular interesting specialised and conserved structures is the mitochondrial respiratory chain, in which 4 complexes play a vital role in the electron transfer, necessary for this energy production process. Also important is the hydrophilic heme protein cytochrome c, which like its "host", also plays other important roles.
\\
Another role in energy production, is its metabolism of lactate, in which mitochondrial L-lactate dehydrogenase (m-L-LDH) oxidises lactate into pyruvate, transforming an otherwise "waste product" into energy supply. In fact, owing to this mitochondria role, lactate appears to in fact be an essential energy source for several tissues. Central to this metabolism is its shuttling within cells, in which mitochondria also plays an important role. This shuttling depends on pH and concentration gradients, which are generate by the mitochondria.
\\
This importance in energy production is supported by their very high motility, as mitochondria are traficked into sites of high demand, transported through the cytoskeleton. These sites of high demand are perinuclear or close to the endoplasmatic reticulum within the cell. On more specialised cells, the mitochondria is enriched on the synaptic area, consistent with the high energetic demands of this location. An extra regulator of this motility is H202, as it induces it. This regulation suggests that the recruitment of mitochondria at sites of elevated ROS production can have protective effects against ROS.
\\
As a signalling hub, mitochondria are vital in calcium homeostasis, as they can store and release Ca2+, in ratios that the cell requires at that time. 



 The cell cycle control machinery must ensure perfect genome duplication and cell division, the basis for self-replication. Calcium-based signaling is a universal mechanism through which extracellular messengers modify the activity of target cells.  Four fundamental, intimately related, and interdependent processes are responsible for cell life: survival, proliferation, differentiation, and death (Danese et al., 2021). Ca2+ is essential in each of these processes, especially the impact of Ca2+ homeostasis on cell death mechanisms (Contreras et al., 2010; Danese et al., 2021; Garbincius and Elrod, 2022). Mitochondrial Ca2+ uptake primarily depends on the mitochondrial Ca2+ uniporter (MCU). MCU-mediated effects drive cell cycle, ATP, and ROS production (Zhao and Pan, 2021). As aforementioned, disturbances of Ca2+ buffering can lead to many neurodegenerative disorders. Therefore, the regulation of Ca2+ by mitochondria should be an important target for the primary and secondary prevention of a wide range of neurodegenerative diseases (Rodríguez et al., 2022).



% ---------------------
% 1. Basic Figure
% ---------------------

% You can also drag the figures to this section and it will automatically include them

\begin{figure}[htbp]
  \centering
  \includegraphics[width=0.8\textwidth]{images/logo/ulisboa-only.pdf}
  \caption{This is a basic figure.}
  \label{fig:basic}
\end{figure}
 

% ---------------------
% 2. Figure with Specific Height and Width
% ---------------------
\begin{figure}[htbp]
  \centering
  \includegraphics[width=5cm]{images/logo/ulisboa-only.pdf}
  \caption{Figure with custom dimensions.}
  \label{fig:customsize}
\end{figure}

% ---------------------
% 3. Side-by-Side Figures (Subfigures)
% ---------------------
\begin{figure}[htbp]
  \centering
  \begin{subfigure}[b]{0.45\textwidth}
    \includegraphics[width=\textwidth]{images/logo/ulisboa-only.pdf}
    \caption{First image}
    \label{fig:sub1}
  \end{subfigure}
  \hfill
  \begin{subfigure}[b]{0.45\textwidth}
    \includegraphics[width=\textwidth]{images/logo/ulisboa-only.pdf}
    \caption{Second image}
    \label{fig:sub2}
  \end{subfigure}
  \caption{Two images side by side}
  \label{fig:sidebyside}
\end{figure}

% ---------------------
% 4. Basic Table
% ---------------------
\begin{table}[htbp]
  \centering
  \begin{tabular}{|c|c|c|}
    \hline
    A & B & C \\
    \hline
    1 & 2 & 3 \\
    4 & 5 & 6 \\
    \hline
  \end{tabular}
  \caption{Basic table}
  \label{tab:basic}
\end{table}

% ---------------------
% 5. Table with Multi-Column Header
% ---------------------
\begin{table}[htbp]
  \centering
  \begin{tabular}{|c|c|c|}
    \hline
    \multicolumn{2}{|c|}{Group 1} & Group 2 \\
    \hline
    X & Y & Z \\
    \hline
    1 & 2 & 3 \\
    \hline
  \end{tabular}
  \caption{Table with merged header}
  \label{tab:mergeheader}
\end{table}

% ---------------------
% 6. Table with Fixed Width Columns
% ---------------------
\begin{table}[htbp]
  \centering
  \begin{tabular}{|>{\raggedright\arraybackslash}p{3cm}|p{5cm}|}
    \hline
    Column A & Column B \\
    \hline
    Text with wrapped lines & Another wrapped cell with long content to demonstrate the fixed column widths. \\
    \hline
  \end{tabular}
  \caption{Table with fixed width columns}
  \label{tab:fixedwidth}
\end{table}


