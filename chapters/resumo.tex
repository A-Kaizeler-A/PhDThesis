% !TEX root = ../PhD Thesis.tex
\chapter*{Resumo}
\addcontentsline{toc}{chapter}{Resumo}
% 1200 a 1500 palavras

 Os grânulos de stress (SGs) são condensados citoplasmáticos sem membrana que se formam de forma rápida e reversível em resposta a uma ampla variedade de estímulos celulares, incluindo hipóxia, privação de nutrientes, infeção viral, ativação de células T, choque térmico e stress oxidativo. Estes condensados estão conservados em todos os seres eucarióticos e resultam da agregação de ARN, complexos de pré-iniciação de tradução interrompidos e proteínas, aproximadamente metade das quais são proteínas de ligação a ARN (RBPs). A sua rápida formação (15 a 30 minutos, dependendo do estímulo) e dissipação, assim como a mobilidade contínua dos constituintes proteicos, revelam uma estrutura altamente dinâmica, sugerindo que os SGs desempenham papéis regulatórios adaptativos essenciais durante o stress celular.
 Inicialmente, os SGs foram descritos como elementos centrais na repressão global da tradução que caracteriza a resposta ao stress. Contudo, estudos subsequentes demonstraram que células incapazes de formar SGs ainda mantêm a inibição traducional global, o que levou à proposta atual de que os SGs atuam de forma seletiva, sequestrando proteínas e ARNs específicos, em vez de atuar unicamente na supressão global. Esta capacidade seletiva permite modular a resposta celular de maneira refinada, ajustando processos como apoptose, senescência e proteólise. Por exemplo, SGs podem reter caspases ativas ou DDX3X, prevenindo a pirocitose e favorecendo a sobrevivência celular; PAI-1, cuja acumulação atrasa a senescência; ou RB1, de forma dependente de RBFOX2. Apesar destes exemplos funcionais, o significado biológico completo destes mecanismos permanece parcialmente obscuro. A tradução contínua de alguns ARNs dentro dos SGs desafia a ideia de que o sequestro implica sempre inibição funcional, sobretudo no que diz respeito aos constituintes transcriptómicos.
 Além da função fisiológica, os SGs têm sido implicados em várias patologias. Mutações em proteínas nucleadoras de SGs, como TIA1, TDP43 ou ATXN2, estão associadas a doenças neurodegenerativas, como a esclerose lateral amiotrófica. Além disso, algumas variantes em DDX3X promovem agregados similares a SGs que comprometem a neurogénese. Contudo, estas associações devem ser interpretadas com cautela, uma vez que tanto os SGs como a fisiopatologia destas doenças dependem fortemente da agregação proteica. Em contexto antiviral, vírus como o da hepatite C e o SARS-CoV-2 interferem com a função de G3BP1, uma das proteínas nucleadoras chave, inibindo a formação de SGs e demonstrando que a modulação destes condensados é explorada pelos vírus para contornar a resposta imunitária.
 No contexto oncológico, células tumorais enfrentam elevados níveis de stress intrínseco e extrínseco, resultantes de hipóxia, privação de nutrientes, inflamação e terapias como quimioterapia ou radioterapia, todos potenciais indutores de SGs. Observaram-se SGs em múltiplos tipos tumorais, sendo propostos como promotores de metastização e resistência terapêutica. Experiências de knockout de G3BP1 e COX-1/2 sugerem que a abolição dos SGs reduz a capacidade metastática e aumenta a sensibilidade aos tratamentos, embora não esteja claro se estes efeitos resultam exclusivamente da inibição dos SGs ou de outras funções independentes das proteínas-alvo. Apenas cerca de 20% da G3BP1 localiza-se nos SGs, reforçando a necessidade de interpretar os resultados com cautela. Ainda assim, devido ao seu papel potencial no sequestro seletivo de ARNs e proteínas, os SGs permanecem de grande interesse em múltiplos contextos fisiológicos e patológicos, com possível relevância terapêutica no cancro.
 A caracterização dos SGs recorre principalmente a duas abordagens distintas: centrifugação diferencial (DC) e marcação de proximidade (PL). A DC isola frações celulares ricas em SGs por separação física, seguida de imunoprecipitação; a PL utiliza uma proteína-alvo para marcar moléculas próximas, permitindo identificar constituintes próximos da proteína de interesse. Ambas dependem de proteínas-chave como G3BP1 para a imunoprecipitação ou marcação. Em termos de constituintes proteicos, DC e PL apresentam forte concordância, isolando proteínas consistentes em diferentes células e condições de stress. 
 No entanto, em termos de constituintes transcriptómicos, os métodos divergem: a DC frequentemente indica que ARNs mais longos são enriquecidos, enquanto a PL não reproduz esse padrão. Este fenómeno pode refletir um efeito biológico (ARNs longos possuem mais locais de ligação para RBPs e promovem nucleação) ou um viés metodológico, uma vez que a DC favorece a precipitação de moléculas maiores devido ao seu peso molecular elevado.
 Para resolver estas discrepâncias e caracterizar de forma precisa a composição transcriptómica dos SGs, realizámos uma análise não enviesada de múltiplos conjuntos de dados humanos, abrangendo diferentes linhas celulares, estímulos de stress e métodos de purificação. Os resultados confirmam que os perfis obtidos por DC são fortemente influenciados pelo método, com o comprimento do ARN a ser um preditor dominante. Após correção deste viés, persistem diferenças entre estudos, mas o enriquecimento em ARNs mitocondriais codificados pelo genoma mitocondrial (mas não pelo nuclear) surge como uma característica robusta, reproduzível também em PL e outros métodos.
 Estes achados permitiram-nos propor um modelo funcional: durante stress, a permeabilidade mitocondrial aumenta, permitindo a libertação de ARN de cadeia dupla (dsRNA) através de canais como VDAC e poros regulados por BAX/BAK. Este dsRNA, gerado naturalmente durante a transcrição do ADN mitocondrial circular, atua como sinal de dano celular, ativando sensores antivirais como MDA5 e RIG-I, posteriormente induzindo inflamação. Postulámos que os SGs sequestram este dsRNA, prevenindo a sua deteção pelos sensores e modulando a resposta inflamatória, funcionando como travões homeostáticos. 
 Para testar esta hipótese, cultivámos células de osteossarcoma U2-OS em condições de stress hiperosmótico (sorbitol), com ou sem inibição da permeabilidade mitocondrial (DIDS). Imunofluorescência para mitocôndria, G3BP1 e dsRNA (anticorpo J2) revelou que o bloqueio da saída de dsRNA impede a formação de SGs e que, durante stress, existe forte colocalização entre dsRNA e SGs. Estes resultados suportam a necessidade de dsRNA para nucleação dos SGs e sugerem o papel destes condensados na modulação da inflamação celular, embora estudos adicionais sejam necessários para confirmação definitiva, visto o anticorpo de dsRNA utilizado marcar todo este tipo de moléculas, e não apenas as provenientes da mitocôndria.
 De forma complementar, definimos usando métodos de machine learning uma assinatura transcriptómica de SGs (SGScore) baseada em níveis de expressão, usando um estudo com células não stressadas, células stressadas sem SGs, e células stressadas com SGs. Esta assinatura permite inferir a presença de SGs em datasets de RNA-seq sem recurso a técnicas de imagem, facilitando o estudo destes em grandes estudos transcriptómicos de cancro, como o The Cancer Genome Atlas. Após validação num outro estudo independente, a SGScore revelou associação com pior prognóstico em quatro tipos de cancro: carcinoma adrenocortical, colangiocarcinoma, mesotelioma e adenocarcinoma gástrico. 
 Comparações com tumores onde a SGScore não possui valor prognóstico mostraram que estes 4 cancros apresentam maior inflamação e características metastáticas, mas menor proliferação basal. Dentro destes 4 tumores, pontuações mais altas na SGS correlacionaram com maior expressão de HK2, que promove abertura de VDAC, e alterações em vias de sinalização como aumento de genes proliferativos e diminuição de genes associados à metastização, sugerindo uma modulação adaptativa da célula em stress.
 Em suma, este trabalho fornece avanços significativos na compreensão dos SGs: (i) demonstra e corrige um viés metodológico crítico na caracterização transcriptómica; (ii) identifica a presença consistente de RNA mitocondrial e propõe um modelo funcional integrando permeabilidade mitocondrial, dsRNA e modulação inflamatória por SGs; (iii) define uma assinatura transcriptómica capaz de prever impacto clínico em cancro, reforçando o potencial translacional desta análise. 
 Assim, os SGs emergem não como simples subprodutos do stress, mas como condensados reguladores ativos, com relevância fisiológica, patológica e terapêutica, representando potenciais alvos para estratégias inovadoras de intervenção em doenças humanas.
 
 

\textbf{Palavras-chave:} Grânulos de stress; proteínas de ligação a ARN; transcriptómica; ARN mitocondrial; marcação de proximidade; centrifugação diferencial; viés de tamanho de ARN; modulação imune; resposta celular ao stress.

