% !TEX root = ../PhD Thesis.tex
\chapter*{Summary}
\addcontentsline{toc}{chapter}{Summary}
% min. 300 words
% 5 keywords

Stress granules (SGs) are dynamic, membraneless cytoplasmic condensates that form in response to diverse stress stimuli, including oxidative, osmotic, and heatshock stress. Composed of RNA, stalled translation initiation complexes, and RNA-binding proteins (RBPs), SGs have been proposed to modulate cellular stress responses by selectively sequestering specific molecules. However, despite extensive study, their precise molecular composition and biological function remain poorly defined. In particular, transcriptomic profiles of SGs differ markedly between studies, largely reflecting methodological discrepancies between differential centrifugation (DC) and proximity labeling (PL) approaches.
To clarify these inconsistencies, we performed a comprehensive reanalysis of publicly available human SG transcriptomes encompassing multiple cell types, stresses, and purification strategies. We found that DC-based datasets are heavily influenced by RNA length, consistent with a physical bias inherent to sedimentation-based separation, which favors precipitation of longer, higher-molecular-weight RNAs. After correcting for this effect, inter-study concordance remained limited, but a reproducible enrichment in mitochondrially encoded RNAs was consistently observed across both DC and PL datasets.
To explore the biological implications of this mitochondrial RNA enrichment, we hypothesized that SGs may function as buffers against inflammation by sequestering mitochondrial double-stranded RNA (dsRNA), which acts as a damage-associated molecular pattern that activates inflammatory antiviral pathways. Supporting this idea, immunofluorescence assays in U2-OS cells showed strong colocalisation of dsRNA with G3BP1-positive SGs under hyperosmotic stress, whereas SG formation was impaired when mitochondrial membrane permeability was blocked. These observations suggest that mitochondrial dsRNA may contribute to SG nucleation and that its sequestration within SGs could prevent aberrant activation of innate immune signaling.
Finally, we derived an SG-associated gene signature from transcriptomic data of unstressed, and stressed cells with and without SG formation. This signature, validated in an independent dataset, revealed prognostic value across several human cancers, including adrenocortical carcinoma, cholangiocarcinoma, mesothelioma, and gastric adenocarcinoma, where higher SG activity correlated with poorer survival. Collectively, our results reveal that SG composition is strongly shaped by methodological bias, that mitochondrial RNA sequestration represents a conserved feature of SGs, and that SG formation may serve as a regulatory “brake” on inflammation with potential implications in cancer biology, acting as putative new therapeutic targets.



\textbf{Keywords:} Stress granules; RNA-binding proteins; transcriptomics; mitochondrial RNA; proximity labelling; differential centrifugation; RNA length bias; immune modulation; cellular stress response.

